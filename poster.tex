%%%%%%%%%%%%%%%%%%%%%%%%%%%%%%%%%%%%%%%%%%%%%%%%%%%%%%%%%%%%%%%%%%%%%%%%%
% sample poster file 
%%%%%%%%%%%%%%%%%%%%%%%%%%%%%%%%%%%%%%%%%%%%%%%%%%%%%%%%%%%%%%%%%%%%%%%%%

%%%%%%%%%%%%%%%%%%%%%%%%%%%%%%%%%%%%%%%%%%%%%%%%%%%%%%%%%%%%%%%%%%%%%%%%%
%% IGL Template
% Last modified: Tue 19 Feb 2013 02:42:39 PM CST
%%%%%%%%%%%%%%%%%%%%%%%%%%%%%%%%%%%%%%%%%%%%%%%%%%%%%%%%%%%%%%%%%%%%%%%%%


%%%%%%%%%%%%%%%%%%%%%%%%%%%%%%%%%%%%%%%%%%%%%%%%%%%%%%%%%%%%%%%%%%%%%%%%%
% documentclass option: pick one:
% "presentation" for powerpoint-like talk,
% "handout" for  printing,
% "trans" for printing onto transparencies
%%%%%%%%%%%%%%%%%%%%%%%%%%%%%%%%%%%%%%%%%%%%%%%%%%%%%%%%%%%%%%%%%%%%%%%%%


\documentclass[leqno,presentation]{beamer}
%\documentclass[leqno, handout]{beamer}
%\documentclass[leqno,trans]{beamer}


%%%%%%%%%%%%%%%%%%%%%%%%%%%%%%%%%%%%%%%%%%%%%%%%%%%%%%%%%%%%%%%%%%%%%%%%%
% beamerposter stuff
%%%%%%%%%%%%%%%%%%%%%%%%%%%%%%%%%%%%%%%%%%%%%%%%%%%%%%%%%%%%%%%%%%%%%%%%%
\usepackage[orientation=landscape, size=a0, scale=1.3]{beamerposter}
\usepackage{pgf}
\usepackage{tikz}
% these packages may be needed 
\usepackage{calc}
\usepackage{times}
\usepackage{type1cm}
\usepackage[latin1]{inputenc}
\usepackage{graphicx}
\usepackage{setspace}
\usepackage{subcaption}
\usepackage{showframe}
\usepackage{amsmath}
%%%%%%%%%%%%%%%%%%%%%%%%%%%%%%%%%%%%%%%%%%%%%%%%%%%%%%%%%%%%%%%%%%%%%%%%%
% end beamerposter stuff
%%%%%%%%%%%%%%%%%%%%%%%%%%%%%%%%%%%%%%%%%%%%%%%%%%%%%%%%%%%%%%%%%%%%%%%%%


% load standard packages

\usepackage{amsmath,amssymb,latexsym, amsthm}
\usepackage[english]{babel}


%\usepackage{hyperref}
%\hypersetup{pdfborderstyle={/S/U/W },pdfborder=0 0 1}

%\usepackage{graphicx}
%\DeclareGraphicsExtensions{.eps, .pdf,.jpg,.png, .tif}

%%%%%%%%%%%%%%%%%%%%%%%%%%%%%%%%%%%%%%%%%%%%%%%%%%%%%%%%%%%%%%%%%%%%%%%%%
% theme
%%%%%%%%%%%%%%%%%%%%%%%%%%%%%%%%%%%%%%%%%%%%%%%%%%%%%%%%%%%%%%%%%%%%%%%%%


%% this seems to work fine

\usetheme{Darmstadt}

%\usetheme{Singapore}


%% beamerposter example uses Berlin, but couldn't get rid of footers 

%\usetheme{Berlin}
%\usetheme{Warsaw}


%\usetheme{PaloAlto}
%\usetheme{AnnArbor}
%\usetheme{CambridgeUS}
%\usetheme{CambridgeUS}
%\usetheme{Berkeley}
%\usetheme{Madrid}


%%%%%%%%%%%%%%%%%%%%%%%%%%%%%%%%%%%%%%%%%%%%%%%%%%%%%%%%%%%%%%%%%%%%%%%%%
%%%%%%%%%%%%%%%%%%%%%%%%%%%%%%%%%%%%%%%%%%%%%%%%%%%%%%%%%%%%%%%%%%%%%%%%%
% begin macros
%%%%%%%%%%%%%%%%%%%%%%%%%%%%%%%%%%%%%%%%%%%%%%%%%%%%%%%%%%%%%%%%%%%%%%%%%
%%%%%%%%%%%%%%%%%%%%%%%%%%%%%%%%%%%%%%%%%%%%%%%%%%%%%%%%%%%%%%%%%%%%%%%%%


% theorem declarations

%\newtheorem{question1}{Question 1}
%\newtheorem{question2}{Question 2}

\theoremstyle{definition}
%\newtheorem{brokenstickproblem}{Broken Stick Problem}




%%%%%%%%%%%%%%%%%%%%%%%%%%%%%%%%%%%%%%%%%%%%%%%%%%%%%%%%%%%%%%%%%%%%%%%%%
%%%%%%%%%%%%%%%%%%%%%%%%%%%%%%%%%%%%%%%%%%%%%%%%%%%%%%%%%%%%%%%%%%%%%%%%%
% end macros
%%%%%%%%%%%%%%%%%%%%%%%%%%%%%%%%%%%%%%%%%%%%%%%%%%%%%%%%%%%%%%%%%%%%%%%%%
%%%%%%%%%%%%%%%%%%%%%%%%%%%%%%%%%%%%%%%%%%%%%%%%%%%%%%%%%%%%%%%%%%%%%%%%%

%%%%%%%%%%%%%%%%%%%%%%%%%%%%%%%%%%%%%%%%%%%%%%%%%%%%%%%%%%%%%%%%%%%%%%%%%
% title/author/date
%%%%%%%%%%%%%%%%%%%%%%%%%%%%%%%%%%%%%%%%%%%%%%%%%%%%%%%%%%%%%%%%%%%%%%%%%

%% size up title and author for poster 

\title{
\veryHuge
Music and Sound Programming using Mathematical Patterns}
  \author{
  \LARGE
  Habeen Chang, Zeyu (Bill) Hu, Zirun (Mike) Lin, Ye Luo*, Alex Yang, Hang Yang*}
  

%% uiuc and igl logos DON'T CHANGE ANYTHING HERE
\institute{
\raisebox{-2ex}{\includegraphics[width=3in]{rice.png}}
%\hspace{4em}
%\raisebox{-2ex}{\includegraphics[width=1in]{igl-logo-small.png}}
% \hspace{2em} 
{\Large\textcolor{blue!50!orange}{ \Huge{Rice Geometry Laboratory}}}
}


\newcommand\textbox[1]{%
  \parbox{.333\textwidth}{#1}%
}
\date{\large \textbox{~\hfill}\textbox{\hfil RGL Open House, April 21,
2017\hfil}\textbox{\hfill (*mentors)} }




%% logo, shows up  in top left corner of each frame
%\logo{%
%\includegraphics[width=0.4in]{igl-logo-small.png}
%}


%%%%%%%%%%%%%%%%%%%%%%%%%%%%%%%%%%%%%%%%%%%%%%%%%%%%%%%%%%%%%%%%%%%%%%%%%
\begin{document}
\begin{frame}

%%%%%%%%%%%%%%%%%%%%%%%%%%%%%%%%%%%%%%%%%%%%%%%%%%%%%%%%%%%%%%%%%%%%%%%%%
% title frame 
%%%%%%%%%%%%%%%%%%%%%%%%%%%%%%%%%%%%%%%%%%%%%%%%%%%%%%%%%%%%%%%%%%%%%%%%%

\begin{block}{}
\titlepage
\end{block}

%%%%%%%%%%%%%%%%%%%%%%%%%%%%%%%%%%%%%%%%%%%%%%%%%%%%%%%%%%%%%%%%%%%%%%%%%
% body of poster
%%%%%%%%%%%%%%%%%%%%%%%%%%%%%%%%%%%%%%%%%%%%%%%%%%%%%%%%%%%%%%%%%%%%%%%%%

\begin{columns}[t]

%%%%%%%%%%%%%%%%%%%%%%%%%%%%%%%%%%%%%%%%%%%%%%%%%%%%%%%%%%%%%%%%%%%%%%%%%
% left column
%%%%%%%%%%%%%%%%%%%%%%%%%%%%%%%%%%%%%%%%%%%%%%%%%%%%%%%%%%%%%%%%%%%%%%%%%

\begin{column}{.25\linewidth}
%%%%%%%%%%%%%%%%%%%%%%%%%%%%%%%%%%%%%%
% left block 1
%%%%%%%%%%%%%%%%%%%%%%%%%%%%%%%%%%%%%%
\begin{block}{Abstract}
This project aims to explore musical sound effects based on the Fourier synthesis of sinusoidal waves with adjustable parameters derived from interesting mathematical patterns.
This project consists of two stages:
\begin{itemize}
\item The first stage involves utilizing the Fourier Transform to decompose given sound waves into sinusoidal functions and perform frequency domain analysis of the waves. 
\item For the second stage, we will incorporate nice mathematical patterns into musical composition using digital technology. 
\item We currently have a programmable digital instrument capable of performing real-time frequency domain analysis as well as pitch/chord generation.
\end{itemize}
\end{block}
%%%%%%%%%%%%%%%%%%%%%%%%%%%%%%%%%%%%%%
% end left block 1
%%%%%%%%%%%%%%%%%%%%%%%%%%%%%%%%%%%%%%

%%%%%%%%%%%%%%%%%%%%%%%%%%%%%%%%%%%%%%
% left block 2
%%%%%%%%%%%%%%%%%%%%%%%%%%%%%%%%%%%%%%
\begin{block}{Fourier Analysis}
\vspace{1ex}
\begin{itemize}
\item Any periodic function can be written as an infinite sum of sinusoidal functions, known as its Fourier Series.
\[ f(x) = \frac{1}{2} a_0 + \sum^\infty_{n = 1}{a_n \cos (nx)} + \sum^\infty_{n = 1}{b_n \sin (nx)}, \text{ where} \]
\[ a_n = \frac{1}{\pi} \int^\pi_{-\pi} f(x) \cos (nx) \, dx, \quad b_n = \frac{1}{\pi} \int^\pi_{-\pi} f(x) \sin (nx) \, dx \]
\item A sinusoidal function is often given by its amplitude at each given time. However, the frequency domain is preferred in digital signal processing and spectrogram analysis.
\item This type of domain shows how the sound wave energy (amplitude) is distributed over a range of frequencies.
%\begin{figure}
%\centering
%\includegraphics[height = 8cm, width = %16cm]{time_frequency_domain.png}
%\end{figure}
\item The Fourier Transform decomposes a function in time domain into a sum of sine wave  components in the frequency domain and plots the frequency of each sine wave against its corresponding amplitude. 
\[ F(k) = \int^\infty_{-\infty} f(x) e^{-2 \pi i k x} dx \]
% \item Furthermore, the Fourier transform outputs complex numbers as the frequency domain allows one to determine both the amplitude and phase of a sound wave.
\end{itemize}
\end{block}
%%%%%%%%%%%%%%%%%%%%%%%%%%%%%%%%%%%%%%
% end left block 2
%%%%%%%%%%%%%%%%%%%%%%%%%%%%%%%%%%%%%%

%%%%%%%%%%%%%%%%%%%%%%%%%%%%%%%%%%%%%%
% left block 3
%%%%%%%%%%%%%%%%%%%%%%%%%%%%%%%%%%%%%%
\begin{block}{Nyquist-Shannon Sampling Theorem}
\begin{itemize}
\vspace{1ex}
\item If a function which has a Fourier series representation has a maximum frequency of B, then it is completely determined by its samples taken at distances \( 1/(2B) \) apart from each other on the horizontal axis. 
\item In digital signal processing, this theorem gives the minimum sampling rate required for the digital signal to accurately represent the analog signal. 
\end{itemize}
\end{block}
%%%%%%%%%%%%%%%%%%%%%%%%%%%%%%%%%%%%%%
% end left block 3
%%%%%%%%%%%%%%%%%%%%%%%%%%%%%%%%%%%%%%
\end{column}

%%%%%%%%%%%%%%%%%%%%%%%%%%%%%%%%%%%%%%%%%%%%%%%%%%%%%%%%%%%%%%%%%%%%%%%%%
% end left column
%%%%%%%%%%%%%%%%%%%%%%%%%%%%%%%%%%%%%%%%%%%%%%%%%%%%%%%%%%%%%%%%%%%%%%%%%


%%%%%%%%%%%%%%%%%%%%%%%%%%%%%%%%%%%%%%%%%%%%%%%%%%%%%%%%%%%%%%%%%%%%%%%%%
% middle column, wide
%%%%%%%%%%%%%%%%%%%%%%%%%%%%%%%%%%%%%%%%%%%%%%%%%%%%%%%%%%%%%%%%%%%%%%%%%

\begin{column}{.45\linewidth}

%%%%%%%%%%%%%%%%%%%%%%%%%%%%%%%%%%%%%%
% center block 1
%%%%%%%%%%%%%%%%%%%%%%%%%%%%%%%%%%%%%%
% \begin{block}{Fourier Series Representation}

% % \[ f(x) = \frac{1}{2} a_0 + \sum^\infty_{n = 1}{a_n \cos (nx)} + \sum^\infty_{n = 1}{b_n \sin (nx)}, \text{ where} \]
% % \[ a_n = \frac{1}{\pi} \int^\pi_{-\pi} f(x) \cos (nx) \, dx, \quad b_n = \frac{1}{\pi} \int^\pi_{-\pi} f(x) \sin (nx) \, dx \]

% \end{block}

% \begin{block}{Fourier Transform}

% \[ F(k) = \int^\infty_{-\infty} f(x) e^{-2 \pi i k x} dx \]

% \end{block}

% \begin{block}{Nyquist-Shannon Sampling Theorem}

% If \( f \in L_1(\mathbb{R}) \) and \( \hat{f} \), the Fourier transform of \( f \), is supported on the interval \( [-B, B] \), then 

% \begin{equation} \label{eq:1}
% f(x) = \sum_{n \in \mathbb{Z}} f\left( \frac{n}{2B} \right) \text{sinc} \left( 2B \left( x - \frac{n}{2B} \right) \right),
% \end{equation}

% where the equality holds in the \( L_2 \) sense, that is, the series in the RHS of \eqref{eq:1} converges to \( f \) in \( L_2(\mathbb{R}) \).

% \qed

% \end{block}

\begin{block}{Spectrograms of Sound Waves}

\begin{figure}[p]
    \centering
	\begin{minipage}{.25 \textwidth}
    \centering
    \includegraphics[width=.4\linewidth]{A_major.PNG}
    \captionof{figure}{A Major chord.}
    \label{tf1}
	\end{minipage}%
    \begin{minipage}{.4 \textwidth}
    \centering
    \includegraphics[width=.8\linewidth]{audioplot.png}
    \captionof{figure}{Composite soundwave.}
    \label{tf2}
	\end{minipage}%
	\begin{minipage}{.3 \textwidth}
    \centering
    \includegraphics[width=.8\linewidth]{spectrogram.png}
    \captionof{figure}{Spectrogram frequency visualization.}
    \label{tf3}
	\end{minipage}
\end{figure}

\end{block}
%%%%%%%%%%%%%%%%%%%%%%%%%%%%%%%%%%%%%%
% end center block 1
%%%%%%%%%%%%%%%%%%%%%%%%%%%%%%%%%%%%%%

%%%%%%%%%%%%%%%%%%%%%%%%%%%%%%%%%%%%%%
% center block 2
%%%%%%%%%%%%%%%%%%%%%%%%%%%%%%%%%%%%%%
\begin{block}{Fourier Transformations of Periodic Functions}

\begin{figure}[p]
    \centering
	\begin{minipage}{.28 \textwidth}
    \centering
    \includegraphics[width=.8\linewidth]{time_frequency_domain.png}
    \captionof{figure}{Decomposition of soundwave into frequency components}
    \label{tf1}
	\end{minipage}%
    \begin{minipage}{.38 \textwidth}
    \centering
    \includegraphics[width=.6\linewidth]{complex_wave.png}
    \captionof{figure}{An example sound wave in the time domain}
    \label{tf2}
	\end{minipage}%
	\begin{minipage}{.35 \textwidth}
    \centering
    \includegraphics[width=.9\linewidth]{fft.png}
    \captionof{figure}{Same wave transformed to the frequency domain}
    \label{tf3}
	\end{minipage}
\end{figure}

\end{block}
%%%%%%%%%%%%%%%%%%%%%%%%%%%%%%%%%%%%%%%%%%
\begin{block}{Fourier Transformations of Aperiodic Functions}

\begin{figure}[p]
    \centering
	\begin{minipage}{.33 \textwidth}
    \centering
    \includegraphics[width=.65\linewidth]{Aperiodic_Time_Domain.png}
    \caption{A 0.2s-long guitar sound.}
    \label{tf1}
	\end{minipage}%
    \begin{minipage}{.33 \textwidth}
    \centering
    \includegraphics[width=.65\linewidth]{Aperiodic_Fourier_Domain.png}
    \captionof{figure}{The Fourier Transform.}
    \label{tf2}
	\end{minipage}
\end{figure}
\end{block}
% end center block 2
%%%%%%%%%%%%%%%%%%%%%%%%%%%%%%%%%%%%%%
\begin{block}{Fourier Synthesis}

\begin{figure}[p]
    \centering
	\begin{minipage}{.25 \textwidth}
    \centering
    \includegraphics[width=.9\linewidth]{squarewave.png}
    \captionof{figure}{Square wave.}
    \label{1}
	\end{minipage}%
	\begin{minipage}{.5 \textwidth}
    \centering
    \includegraphics[width=.9\linewidth]{synthesis1.png}
    \captionof{figure}{2 terms of the Fourier sine series.}
    \label{2}
	\end{minipage}%
    \begin{minipage}{.25 \textwidth}
    \centering
    \includegraphics[width=.9\linewidth]{synthesis2.png}
    \captionof{figure}{10 terms of the Fourier sine series.}
    \label{3}
	\end{minipage}%
\end{figure}

\end{block}

%%%%%%%%%%%%%%%%%%%%%%%%%%%%%%%%%%%%%%
% center block 3
%%%%%%%%%%%%%%%%%%%%%%%%%%%%%%%%%%%%%%
\begin{columns}
\begin{column}{.49\linewidth}
\begin{block}{Sampling Visualization}

\begin{figure}
    \centering
   	\includegraphics[width=.8\linewidth]{samplingpic2.jpg}
    \caption{Sampling}
    \label{sampling}
\end {figure}

\end{block}
\end{column}

\begin{column}{.49\linewidth}
\begin{block}{Derivative of a Sound Wave}
\begin{figure}
    \centering
   	\includegraphics[width=.8\linewidth]{coolfunction2.jpg}
    \caption{The original function(blue) and its derivative(orange)}
    \label{sderivative}
\end {figure}
\end{block}
\end{column}
\end{columns}
%%%%%%%%%%%%%%%%%%%%%%%%%%%%%%%%%%%%%%
% end center block 3
%%%%%%%%%%%%%%%%%%%%%%%%%%%%%%%%%%%%%%

\end{column}

%%%%%%%%%%%%%%%%%%%%%%%%%%%%%%%%%%%%%%%%%%%%%%%%%%%%%%%%%%%%%%%%%%%%%%%%%
% end middle column, wide
%%%%%%%%%%%%%%%%%%%%%%%%%%%%%%%%%%%%%%%%%%%%%%%%%%%%%%%%%%%%%%%%%%%%%%%%%


%%%%%%%%%%%%%%%%%%%%%%%%%%%%%%%%%%%%%%%%%%%%%%%%%%%%%%%%%%%%%%%%%%%%%%%%%
% right column, narrow 
%%%%%%%%%%%%%%%%%%%%%%%%%%%%%%%%%%%%%%%%%%%%%%%%%%%%%%%%%%%%%%%%%%%%%%%%%

\begin{column}{.25\linewidth}

%%%%%%%%%%%%%%%%%%%%%%%%%%%%%%%%%%%%%%
% right block 1
%%%%%%%%%%%%%%%%%%%%%%%%%%%%%%%%%%%%%%
\begin{block}{Results}
\vspace{1ex}
\begin{itemize}
\item The derivative of a sinusoidal function will produce a sound that is either more or less bright than the sound of the original function. 
\item The derivative of the function $f(t)=\sum_{n=1}^{\infty} \frac{1}{n^2}\sin{(nwt)}$\\ has the same frequency as f(t) but with a phase shift and larger amplitude. The audible result is a significantly brighter tone produced.
\item When a periodic function is written as a sum of sinusoidal functions, a phase shift in each of the individual functions corresponds to a time delay for that particular tone being played.
\item Phase shifts of any magnitude are shown to be inaudible by the human ear, demonstrated by modulating the function $f(t)=\sin{wt}+\sin{(2wt+\phi)}$\\
\end{itemize}
\end{block}
%%%%%%%%%%%%%%%%%%%%%%%%%%%%%%%%%%%%%%
% end right block 1
%%%%%%%%%%%%%%%%%%%%%%%%%%%%%%%%%%%%%%

%%%%%%%%%%%%%%%%%%%%%%%%%%%%%%%%%%%%%%
% right block 2
%%%%%%%%%%%%%%%%%%%%%%%%%%%%%%%%%%%%%%
\begin{block}{Further Work}
\vspace{1ex}
\begin{itemize}
\item We will continue to experiment with the mathematical properties of sound waves using computer software.
\item Additionally, we also hope to utilize a digital synthesizer and amplifier to generate unique sounds through techniques such as resonance and distortion.
\item We will incorporate nice patterns into musical composition by following mathematical rules behind why certain note progressions and chords are harmonically pleasing. 
%Sound experiments will be done based on aural experience as well as mathematical pitch analysis. 
\item Students and faculty at the Shepherd School of Music at Rice will be consulted for their necessary expertise. 
\item The end result will be a short piece of electronic music composed entirely using Fourier analysis.
\end{itemize}
\end{block}
%%%%%%%%%%%%%%%%%%%%%%%%%%%%%%%%%%%%%%
% end right block 2
%%%%%%%%%%%%%%%%%%%%%%%%%%%%%%%%%%%%%%

%%%%%%%%%%%%%%%%%%%%%%%%%%%%%%%%%%%%%%
% right block 3
%%%%%%%%%%%%%%%%%%%%%%%%%%%%%%%%%%%%%%
\begin{block}{References}
\vspace{1ex}
\begin{itemize}
\item"Fourier Series." Fourier Series -from Wolfram MathWorld. Wolfram, n.d. Web. 10 Apr. 2017.
\item"Time and Frequency Domain." Time and Frequency Domain. Erzetich, n.d. Web. 10 Apr. 2017.
\item"Documentation." Practical Introduction to Frequency-Domain Analysis - MATLAB and Simulink Example. Mathworks, n.d. Web. 10 Apr. 2017.
\item Roberts, Gareth E. From Music to Mathematics: Exploring the Connections. Baltimore, MD: Johns Hopkins UP, 2016. Print.
\item Lerman, Gilad, and Notes For Math 5467. The Shannon Sampling Theorem and Its Implications (n.d.): 1-2. Math.ums. University of Minnesota. Web. 3 Mar. 2017.
%Hanna, J. Ray., and John H. Rowland. Fourier Series, Transforms, and Boundary Value Problems. Mineola, NY: Dover Publications, 2008. Print.
%\vspace{32ex} %comment out when text is added
\end{itemize}
\end{block}
%%%%%%%%%%%%%%%%%%%%%%%%%%%%%%%%%%%%%%
% end right block 3
%%%%%%%%%%%%%%%%%%%%%%%%%%%%%%%%%%%%%%

\end{column}
%%%%%%%%%%%%%%%%%%%%%%%%%%%%%%%%%%%%%%%%%%%%%%%%%%%%%%%%%%%%%%%%%%%%%%%%%
% end right column
%%%%%%%%%%%%%%%%%%%%%%%%%%%%%%%%%%%%%%%%%%%%%%%%%%%%%%%%%%%%%%%%%%%%%%%%%

\end{columns}
%%%%%%%%%%%%%%%
%Funding Acknowledgements: Make sure you get this right before sending to printers
%%%%%%%%%%%%%%%
  \begin{block}{}
   \begin{center}
   %These posters are made with the support of University of Illinois at Urbana-Champaign Public Engagement Office
  \end{center}
  \end{block}

